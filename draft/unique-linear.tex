\documentclass{llncs}

\usepackage{amsmath}
\usepackage{amsfonts}
\usepackage[utf8]{inputenc}
\usepackage{mathpartir}

\newcommand\kw{\textsf}
\newcommand\meta{\textit}

\newcommand\FIX[2]{\kw{fix}\,#1{:}#2.}
\newcommand\LAM[2]{\lambda#1{:}#2.}
\newcommand\APP[1]{#1\,}
\newcommand\LET[2]{\kw{let}\,#1=#2\,\kw{in}\,}
\newcommand\REC[2][{}]{\kw{rec}^{#1}\{#2\}}
\newcommand\GET[1]{#1.}
\newcommand\UPD[4][{}]{#2^{#1}\{#3=#4\}}

\newcommand\IF[2]{\kw{if}\,#1\,#2\,}

\newcommand\OVERRIDE\rhd

\newcommand\JOINK{\meta{join}}
\newcommand\SPLITK{\meta{meet}}

\newcommand\JOIN[3]{\JOINK (#1, #2, #3)}
\newcommand\SPLIT[3]{\SPLITK (#1, #2, #3)}

\newcommand\UN{\kw{un}}
\newcommand\LIN{\kw{lin}}

\newcommand\NOTHING{\bot}
\newcommand\READ{\kw{R}}
\newcommand\WRITE{\kw{W}}

\newcommand\TVAR{\alpha}
\newcommand\PVAR{\pi}
\newcommand\NUM{\kw{num}}

\newcommand\SUB{\le}

\newcommand\FREE{\meta{free}}
\newcommand\COPY{\meta{copy}}

%%% Local Variables: 
%%% mode: latex
%%% TeX-master: "unique-linear"
%%% End: 


\begin{document}
\section{First Attempt: Uniqueness and Linearity}
\label{sec:first-attempt:-uniq}


Syntax
\begin{align*}
  e &::= x \mid \LAM xte \mid \APP ee \mid \LET xee \mid \REC{ \overline{a=e}} \mid \GET ea \mid \UPD eae \\
  \ell &::= \LIN \mid \UN & \UN & \sqsubset \LIN \\
  u &::= \READ \mid \WRITE & \WRITE & \sqsubset \READ \\
  t &::= b\,\ell\, u \\
  b &::= t \to t \mid \REC{ \overline{a : t}}
\end{align*}
Every type has two attributes, linearity $\ell$ and uniqueness $u$.

An environment $A$ is a partial mapping from variables to types. Environment overriding $A=A'\OVERRIDE A''$ is defined as the pointwise extension of a ternary relation on types. We write $x:t\,-$ to indicate that no binding for $x$ exists. 
\begin{mathpar}
  \inferrule{}{b\,\UN\,\READ  =  b\,\UN\,\READ   \OVERRIDE' b\,\UN\,\READ}

  \inferrule{}{b\,\UN\,\WRITE  =  b\,\UN\,\WRITE  \OVERRIDE b\,-}
  \\
  \inferrule{b = b' \OVERRIDE' b''}
  {b\,\UN\,\WRITE  =  b'\,\UN\,\READ   \OVERRIDE' b''\, \UN\,\WRITE}

  \inferrule{t = t' \OVERRIDE' t''}{t = t' \OVERRIDE t''}
  \\
  \inferrule{}{t\,\LIN\,u           =  t\,\LIN\,u           \OVERRIDE t\,-}

  \inferrule{}{t\,\LIN\,u           =  t\,-                     \OVERRIDE t\, \LIN\,u}
  \\
  \inferrule{
    t_i = t_i' \OVERRIDE' t_i''
  }{
    \REC{ a_1:t_1, \dots } =     \REC{ a_1:t_1', \dots } \OVERRIDE'     \REC{ a_1:t_1'', \dots }
  }

  \inferrule{}{t_1\to t_2 = t_1 \to t_2 \OVERRIDE' t_1 \to t_2}
\end{mathpar}
For $n>2$ define $A= A_1 \OVERRIDE A_2 \OVERRIDE \dots \OVERRIDE A_n$ by $A= A_1 \OVERRIDE A_1'$ and $A_1' = A_2 \OVERRIDE \dots \OVERRIDE A_n$. 

\newpage
If any record component is linear, then the record must be linear, too.
If any of the components has write premission, then the record must have it, too.
\begin{mathpar}
  \inferrule{
    A = A_1 \OVERRIDE \dots \OVERRIDE A_n \\
    A_i \vdash e_i : t_i\,\ell_i\,u_i \\
    \ell_i \sqsubseteq \ell \\
    u \sqsubseteq u_i
  }{
    A \vdash \REC{ \overline{a_i=e_i}} : \REC{ \overline{a_i : t_i}}\,\ell\, u
  }
\end{mathpar}
Reading from a record is a use of the record: If the record is linear, then there can be only one use. The fields that are dropped  must not be linear. Thus, a linear record can have at most a single linear field and no other field can be accessed. (That seems overly restrictive, but could be amended by a record elimination that reads more than one field at a time.) 

Uniqueness of the record does not matter for reading. The attributes of the field's type remain as they are, if the record has write permission. (If the record has only read permission, then none of the fields can have write permission, so the attributes remain the same, too.)
\begin{mathpar}
  \inferrule{
    A \vdash e : \REC{ a:t \mid r}\,\ell\,u \\
    \UN (r)
  }{
    A \vdash \GET ea : t
  }
\end{mathpar}
To write to a record, we need write permission. The type of the overwritten field must be unrestricted because the old contents is dropped. (It would be good to have an update operation that works on linear fields.)
\begin{mathpar}
  \inferrule{
    A_0 \vdash e_0 : \REC{ a:t \mid r}\,\ell\,\WRITE \\
    A_1 \vdash e_1 : t \\
    \UN (t) \\
    A = A_0 \OVERRIDE A_1
  }{
    A \vdash \UPD {e_0}a{e_1} : \REC{ a:t \mid r}\,\ell\,u
  }
\end{mathpar}

\begin{mathpar}
  \inferrule{
  }{
    A, x: t \vdash x:t
  }
\end{mathpar}
Uniqueness does not matter for a function. However, a function that captures a linear variable or a variable with write permission must be linear. A function that captures a record with read permission can be influenced by subsequent writes to that record. Thus the following combinations are possible.
\begin{itemize}
\item $y:t \in A$ is readonly, which means that no future writes through $y$ are possible. Readonly variables impose no restriction on the function type, but currently we have no notion of readonly.
\item $y:t \in A$ needs write permission to typecheck $e$. This requires $\ell = \LIN$, so that writes are only executed at  most once. If $\ell \ne\LIN$, then $y$ needs to be copied each time the closure is invoked.
\item $y:t \in A$ is linear requires that $\ell = \LIN$, so that the linear stuff is used exactly once.
\end{itemize}
\begin{mathpar}
  \inferrule{
    A, x:t \vdash e : t'
  }{
    A \vdash \LAM x t e : (t \to t')\,\ell\,u
  }
\end{mathpar}
Function application has no particular restrictions, it seems.
\begin{mathpar}
  \inferrule{
    A_0 \vdash e_0 : (t_1 \to t_2)\, \ell\,u \\
    A_1 \vdash e_1 : t_1 \\
    A = A_0 \OVERRIDE A_1
  }{
    A \vdash \APP{e_0}e_1 : t_2
  }
\end{mathpar}

\clearpage{}

\section{Second Attempt: Uniqueness Only}
\label{sec:second-attempt:-uniq}

Syntax
\begin{align*}
  e &::= x \mid \LAM xte \mid \APP ee \mid \LET xee  \mid c \mid e+e \mid \IF eee\\
  &\mid \REC{ \overline{a=e}} \mid \GET ea \mid \UPD[u] eae \\
  u,v &::= \READ \mid \WRITE & \WRITE & \sqsubset \READ \\
  s,t &::= \NUM \mid s \to t \mid \REC[u]{ \overline{a : t}}
\end{align*}
Only record types have a uniqueness attribute $u$.
The $u$ attribute on the update operation determines whether it is destructive ($u=\WRITE$) or copying ($u=\READ$).

\begin{mathpar}
  \inferrule{
    A_2 \vdash e_0 : \REC[u]{ a_1:t \mid r} \dashv A_1 \\
    A_1 \vdash e_1 : t \dashv A_0 
  }{
    A_2 \vdash
    \UPD[u] {e_0}{a_1}{e_1} : \REC[v]{ a_1:t \mid r}
    \dashv A_0
  }

  \inferrule{
    A_1 \vdash e : \REC[\READ]{ a:t \mid r}
    \dashv A_0
  }{
    A_1 \vdash \GET ea : t
    \dashv A_0
  }
  
  \inferrule{
    A_{i+1} \vdash e_i : t_i \dashv A_i \\
  }{
    A_{n+1} \vdash \REC{ \overline{a_i=e_i}} : \REC[u]{ \overline{a_i : t_i}}
    \dashv A_0
  }

  \inferrule{
    t_0 = t_1 \OVERRIDE t
  }{
    A, x: t_1 \vdash x : t \dashv A, x:t_0
  }

    \inferrule{
    A_2 \vdash e_0 : (t_1 \to t_2) \dashv A_1 \\
    A_1 \vdash e_1 : t_1 \dashv A_0
  }{
    A_2 \vdash
    \APP{e_0}e_1 : t_2
    \dashv A_0
  }

    \inferrule{
      A_1, x:t_1 \vdash
      e : t
      \dashv A_0^{X:\READ}, x:t_0 \\
      X = \FREE (\LAM x {t_0} e)
  }{
    A_0^{X:\READ} \vdash
    \LAM x {t_0} e : (t_0 \to t)
    \dashv A_0
  }

  \inferrule{
    A_4 \vdash e_0 : \NUM \dashv A_3 \\
    A_1 \vdash e_1 : t_1 \dashv A_0 \\
    A_2 \vdash e_2 : t_2 \dashv A_0 \\
    t_1, t_2 \sqsubseteq t_3 \\
    A_1, A_2 \sqsubseteq A_3
  }{
    A_4 \vdash \IF {e_0}{e_1}{e_2} : t_3 \dashv A_0
  }
\end{mathpar}

Operationally, the lambda takes a deep copy of all free variables.
Thus, it must not propagate write permission to the left.
As the lambda may be invoked multiple times, the free variables must
not have write permission.
The notation $A^{X:\READ}$ means that all permissions in the types of the
variables in $X$ are set to $\READ$.

\clearpage{}

\section{Third Attempt: Field Effects}
\label{sec:third-attempt:-field}

Effects attached to single record labels
\begin{align*}
  P, Q &\subseteq \{ \READ, \WRITE \}
\end{align*}

\begin{figure}[tp]
  \begin{mathpar}
    \inferrule{
      A = A_1 \OVERRIDE \dots \OVERRIDE A_n \\
      A_i \vdash e_i : t_i }{ A \vdash \REC{ \overline{a_i=e_i}} :
      \REC{ \overline{a_i :^{P_i} t_i}} }

    \inferrule { A \vdash e : \REC{ a:^\READ t \mid r}
      % \\ \READ \in P
    } { A \vdash \GET ea : t }

    \inferrule{
      A_0 \vdash e_0 : \REC{ a:^\WRITE t \mid r} \\
      A_1 \vdash e_1 : t \\
      % \WRITE \in P \\
      A = A_0 \OVERRIDE A_1 }{ A \vdash \UPD {e_0}a{e_1} : \REC{ a:^P
        t \mid r} }
    \\
    \inferrule{ }{ A, x: t \vdash x: t }

    \inferrule{
      A_0, x:t \vdash e : t' \\
      A = A_0^* }{ A \vdash \LAM x t e : (t \to t') }

  \inferrule{
    A_0 \vdash e_0 : (t_1 \to t_2) \\
    A_1 \vdash e_1 : t_1 \\
    A = A_0 \OVERRIDE A_1
  }{
    A \vdash \APP{e_0}e_1 : t_2
  }

  \inferrule
  { A_1 \vdash e_1 : t_1 \\
    A_2, x : t_1 \vdash e_2 : t_2 \\
    A = A_1 \OVERRIDE A_2
  }
  { A \vdash \LET x{e_1} e_2 : t}
  \end{mathpar}
  \caption{Propagation rules}
  \label{fig:field-effects}
\end{figure}
Function closures have ``infinite'' splitting $(a :^P t)^*$ into $(a :^Q t)$.
\begin{itemize}
\item $P \subseteq \{\READ\}$ propagate $Q = \READ$, no action required
\item $\WRITE \in P$: propagate $Q = \READ$, insert copy into closure
\end{itemize}

Function application has no particular restrictions, it seems.
Let improves on lambda combined with app because infinite splitting is avoided.

Look at combining $a :^O t \OVERRIDE a:^P t$ upwards to $a : ^Q t$:
$Q= O \OVERRIDE P$
\begin{itemize}
\item if $O=\emptyset$ or $P=\emptyset$ (field only used in one branch), then $Q = O \cup P$, no
  action required
\item if $O\subseteq \{\READ\}$: $Q = O \cup P$,
  no further action required as potential writing in $P$ happens after reading
\item if $\WRITE \in O$, $P = \{\READ\}$: two choices, not clear
  which is better
  \begin{itemize}
  \item $Q = \READ$ insert copy, propagate left to the write operation
  \item $Q = \WRITE$ insert copy, propagate right to the read operation
  \end{itemize}
\item if $\WRITE\in O$, $\WRITE \in P$: $Q = \WRITE$ insert copy,
  propagate left (or right)
\end{itemize}

Combining environments: pointwise

Combining records
\begin{mathpar}
  \inferrule
  {}
  {
    \REC{ {a :^{Q_a} t_a}, {b :^{Q_b} t_b}}
    =
    \REC{ {a :^{O_a} t_a}, {b :^{O_b} t_b}}
    \OVERRIDE
    \REC{ {a :^{P_a} t_a}, {b :^{P_b} t_b}}
  }
\end{mathpar}

\clearpage
\begin{mathpar}
  \begin{array}{|c|c||c|c||c|c|l}
    Q_a&Q_b&O_a&O_b&P_a&P_b& \\
    \hline
    P_a &\emptyset& \emptyset & \emptyset & P_a & \emptyset & \\
    O_a & \emptyset& O_a & \emptyset & \emptyset & \emptyset & \\
    P_a\cup\READ   &\emptyset& \subseteq\READ & \emptyset & P_a & \emptyset & \\
    \READ&\READ& \WRITE\in & \emptyset & \READ & \emptyset
                       &  \text{copy, propagate left to write}\\
    \WRITE &\READ& \WRITE\in & \emptyset & \WRITE\in & \emptyset
                       &\text{copy, propagate left to write} \\
  \end{array}

  \begin{array}{|c|c||c|c||c|c|l}
    Q_a&Q_b&O_a&O_b&P_a&P_b& \\
    \hline
    P_a & P_b & \emptyset & \emptyset & P_a & P_b & \\
    O_a & P_b & O_a & \emptyset & \emptyset & P_b & \\
    P_a\cup\READ   &P_b& \subseteq\READ & \emptyset & P_a & P_b & \\
    \READ&P_b\cup\READ& \WRITE\in & \emptyset & \READ & P_b
                       &  \text{copy, propagate left to write}\\
    \WRITE & P_b\cup\READ & \WRITE\in & \emptyset & \WRITE\in & P_b
                       &\text{copy, propagate left to write} \\
  \end{array}

  \begin{array}{|c|c||c|c||c|c|l}
    Q_a&Q_b&O_a&O_b&P_a&P_b& \\
    \hline
    P_a & O_b & \emptyset & O_b & P_a & \emptyset & \\
    O_a & O_b & O_a & O_b & \emptyset & \emptyset & \\
    P_a\cup\READ  & O_b & \subseteq\READ & O_b & P_a & \emptyset & \\
    \READ&O_b \cup\READ& \WRITE\in & O_b & \READ & \emptyset
                       &  \text{copy, propagate left to write}\\
    \WRITE &O_b\cup\READ& \WRITE\in & O_b & \WRITE\in & \emptyset
                           &\text{copy, propagate left to write} \\
  \end{array}

\begin{array}{|c|c||c|c||c|c|l}
    Q_a&Q_b&O_a&O_b&P_a&P_b& \\
    \hline
    P_a &\emptyset& \emptyset & \emptyset & P_a & \emptyset & \\
    O_a & \emptyset& O_a & \emptyset & \emptyset & \emptyset & \\
    P_a\cup\READ   &\emptyset& \subseteq\READ & \emptyset & P_a & \emptyset & \\
    \READ&\READ& \WRITE\in & \emptyset & \READ & \emptyset
                       &  \text{copy, propagate left to write}\\
    \WRITE &\READ& \WRITE\in & \emptyset & \WRITE\in & \emptyset
                       &\text{copy, propagate left to write} \\
  \end{array}
\end{mathpar}

\end{document}
