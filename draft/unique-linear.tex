\documentclass{llncs}

\usepackage{amsmath}
\usepackage{amsfonts}
\usepackage[utf8]{inputenc}
\usepackage{mathpartir}

\newcommand\kw{\textsf}
\newcommand\meta{\textit}

\newcommand\FIX[2]{\kw{fix}\,#1{:}#2.}
\newcommand\LAM[2]{\lambda#1{:}#2.}
\newcommand\APP[1]{#1\,}
\newcommand\LET[2]{\kw{let}\,#1=#2\,\kw{in}\,}
\newcommand\REC[2][{}]{\kw{rec}^{#1}\{#2\}}
\newcommand\GET[1]{#1.}
\newcommand\UPD[4][{}]{#2^{#1}\{#3=#4\}}

\newcommand\IF[2]{\kw{if}\,#1\,#2\,}

\newcommand\OVERRIDE\rhd

\newcommand\JOINK{\meta{join}}
\newcommand\SPLITK{\meta{meet}}

\newcommand\JOIN[3]{\JOINK (#1, #2, #3)}
\newcommand\SPLIT[3]{\SPLITK (#1, #2, #3)}

\newcommand\UN{\kw{un}}
\newcommand\LIN{\kw{lin}}

\newcommand\NOTHING{\bot}
\newcommand\READ{\kw{R}}
\newcommand\WRITE{\kw{W}}

\newcommand\TVAR{\alpha}
\newcommand\PVAR{\pi}
\newcommand\NUM{\kw{num}}

\newcommand\SUB{\le}

\newcommand\FREE{\meta{free}}
\newcommand\COPY{\meta{copy}}

%%% Local Variables: 
%%% mode: latex
%%% TeX-master: "unique-linear"
%%% End: 


\begin{document}
\section{First Attempt: Uniqueness and Linearity}
\label{sec:first-attempt:-uniq}


Syntax
\begin{align*}
  e &::= x \mid \LAM xte \mid \APP ee \mid \LET xee \mid \REC{ \overline{a=e}} \mid \GET ea \mid \UPD eae \\
  \ell &::= \LIN \mid \UN & \UN & \sqsubset \LIN \\
  u &::= \READ \mid \WRITE & \WRITE & \sqsubset \READ \\
  t &::= b\,\ell\, u \\
  b &::= t \to t \mid \REC{ \overline{a : t}}
\end{align*}
Every type has two attributes, linearity $\ell$ and uniqueness $u$.

An environment $A$ is a partial mapping from variables to types. Environment overriding $A=A'\OVERRIDE A''$ is defined as the pointwise extension of a ternary relation on types. We write $x:t\,-$ to indicate that no binding for $x$ exists. 
\begin{mathpar}
  \inferrule{}{b\,\UN\,\READ  =  b\,\UN\,\READ   \OVERRIDE' b\,\UN\,\READ}

  \inferrule{}{b\,\UN\,\WRITE  =  b\,\UN\,\WRITE  \OVERRIDE b\,-}
  \\
  \inferrule{b = b' \OVERRIDE' b''}
  {b\,\UN\,\WRITE  =  b'\,\UN\,\READ   \OVERRIDE' b''\, \UN\,\WRITE}

  \inferrule{t = t' \OVERRIDE' t''}{t = t' \OVERRIDE t''}
  \\
  \inferrule{}{t\,\LIN\,u           =  t\,\LIN\,u           \OVERRIDE t\,-}

  \inferrule{}{t\,\LIN\,u           =  t\,-                     \OVERRIDE t\, \LIN\,u}
  \\
  \inferrule{
    t_i = t_i' \OVERRIDE' t_i''
  }{
    \REC{ a_1:t_1, \dots } =     \REC{ a_1:t_1', \dots } \OVERRIDE'     \REC{ a_1:t_1'', \dots }
  }

  \inferrule{}{t_1\to t_2 = t_1 \to t_2 \OVERRIDE' t_1 \to t_2}
\end{mathpar}
For $n>2$ define $A= A_1 \OVERRIDE A_2 \OVERRIDE \dots \OVERRIDE A_n$ by $A= A_1 \OVERRIDE A_1'$ and $A_1' = A_2 \OVERRIDE \dots \OVERRIDE A_n$. 

\newpage
If any record component is linear, then the record must be linear, too.
If any of the components has write premission, then the record must have it, too.
\begin{mathpar}
  \inferrule{
    A = A_1 \OVERRIDE \dots \OVERRIDE A_n \\
    A_i \vdash e_i : t_i\,\ell_i\,u_i \\
    \ell_i \sqsubseteq \ell \\
    u \sqsubseteq u_i
  }{
    A \vdash \REC{ \overline{a_i=e_i}} : \REC{ \overline{a_i : t_i}}\,\ell\, u
  }
\end{mathpar}
Reading from a record is a use of the record: If the record is linear, then there can be only one use. The fields that are dropped  must not be linear. Thus, a linear record can have at most a single linear field and no other field can be accessed. (That seems overly restrictive, but could be amended by a record elimination that reads more than one field at a time.) 

Uniqueness of the record does not matter for reading. The attributes of the field's type remain as they are, if the record has write permission. (If the record has only read permission, then none of the fields can have write permission, so the attributes remain the same, too.)
\begin{mathpar}
  \inferrule{
    A \vdash e : \REC{ a:t \mid r}\,\ell\,u \\
    \UN (r)
  }{
    A \vdash \GET ea : t
  }
\end{mathpar}
To write to a record, we need write permission. The type of the overwritten field must be unrestricted because the old contents is dropped. (It would be good to have an update operation that works on linear fields.)
\begin{mathpar}
  \inferrule{
    A_0 \vdash e_0 : \REC{ a:t \mid r}\,\ell\,\WRITE \\
    A_1 \vdash e_1 : t \\
    \UN (t) \\
    A = A_0 \OVERRIDE A_1
  }{
    A \vdash \UPD {e_0}a{e_1} : \REC{ a:t \mid r}\,\ell\,u
  }
\end{mathpar}

\begin{mathpar}
  \inferrule{
  }{
    A, x: t \vdash x:t
  }
\end{mathpar}
Uniqueness does not matter for a function. However, a function that captures a linear variable or a variable with write permission must be linear. A function that captures a record with read permission can be influenced by subsequent writes to that record. Thus the following combinations are possible.
\begin{itemize}
\item $y:t \in A$ is readonly, which means that no future writes through $y$ are possible. Readonly variables impose no restriction on the function type, but currently we have no notion of readonly.
\item $y:t \in A$ needs write permission to typecheck $e$. This requires $\ell = \LIN$, so that writes are only executed at  most once. If $\ell \ne\LIN$, then $y$ needs to be copied each time the closure is invoked.
\item $y:t \in A$ is linear requires that $\ell = \LIN$, so that the linear stuff is used exactly once.
\end{itemize}
\begin{mathpar}
  \inferrule{
    A, x:t \vdash e : t'
  }{
    A \vdash \LAM x t e : (t \to t')\,\ell\,u
  }
\end{mathpar}
Function application has no particular restrictions, it seems.
\begin{mathpar}
  \inferrule{
    A_0 \vdash e_0 : (t_1 \to t_2)\, \ell\,u \\
    A_1 \vdash e_1 : t_1 \\
    A = A_0 \OVERRIDE A_1
  }{
    A \vdash \APP{e_0}e_1 : t_2
  }
\end{mathpar}

\clearpage{}

\section{Second Attempt: Uniqueness Only}
\label{sec:second-attempt:-uniq}

Syntax
\begin{align*}
  e &::= x \mid \LAM xte \mid \APP ee \mid \LET xee  \mid c \mid e+e \mid \IF eee\\
  &\mid \REC{ \overline{a=e}} \mid \GET ea \mid \UPD[u] eae \\
  u,v &::= \READ \mid \WRITE & \WRITE & \sqsubset \READ \\
  s,t &::= \NUM \mid s \to t \mid \REC[u]{ \overline{a : t}}
\end{align*}
Only record types have a uniqueness attribute $u$.
The $u$ attribute on the update operation determines whether it is destructive ($u=\WRITE$) or copying ($u=\READ$).

\begin{mathpar}
  \inferrule{
    A_2 \vdash e_0 : \REC[u]{ a_1:t \mid r} \dashv A_1 \\
    A_1 \vdash e_1 : t \dashv A_0 
  }{
    A_2 \vdash
    \UPD[u] {e_0}{a_1}{e_1} : \REC[v]{ a_1:t \mid r}
    \dashv A_0
  }

  \inferrule{
    A_1 \vdash e : \REC[\READ]{ a:t \mid r}
    \dashv A_0
  }{
    A_1 \vdash \GET ea : t
    \dashv A_0
  }
  
  \inferrule{
    A_{i+1} \vdash e_i : t_i \dashv A_i \\
  }{
    A_{n+1} \vdash \REC{ \overline{a_i=e_i}} : \REC[u]{ \overline{a_i : t_i}}
    \dashv A_0
  }

  \inferrule{
    t_0 = t_1 \OVERRIDE t
  }{
    A, x: t_1 \vdash x : t \dashv A, x:t_0
  }

    \inferrule{
    A_2 \vdash e_0 : (t_1 \to t_2) \dashv A_1 \\
    A_1 \vdash e_1 : t_1 \dashv A_0
  }{
    A_2 \vdash
    \APP{e_0}e_1 : t_2
    \dashv A_0
  }

    \inferrule{
      A_1, x:t_1 \vdash
      e : t
      \dashv A_0^{X:\READ}, x:t_0 \\
      X = \FREE (\LAM x {t_0} e)
  }{
    A_0^{X:\READ} \vdash
    \LAM x {t_0} e : (t_0 \to t)
    \dashv A_0
  }

  \inferrule{
    A_4 \vdash e_0 : \NUM \dashv A_3 \\
    A_1 \vdash e_1 : t_1 \dashv A_0 \\
    A_2 \vdash e_2 : t_2 \dashv A_0 \\
    t_1, t_2 \sqsubseteq t_3 \\
    A_1, A_2 \sqsubseteq A_3
  }{
    A_4 \vdash \IF {e_0}{e_1}{e_2} : t_3 \dashv A_0
  }
\end{mathpar}

Operationally, the lambda takes a deep copy of all free variables.
Thus, it must not propagate write permission to the left.
As the lambda may be invoked multiple times, the free variables must
not have write permission.
The notation $A^{X:\READ}$ means that all permissions in the types of the
variables in $X$ are set to $\READ$.

\clearpage{}

\section{Third Attempt: Field Effects}
\label{sec:third-attempt:-field}

\begin{figure}[tp]
  \begin{align*}
    e,f,g,r
    &::= x \mid \LAM xte \mid \APP fe \mid \FIX xtf \\
    & \mid \LET xef  \mid c \mid f+g \mid \IF efg\\
    &\mid \REC{ \overline{a=e}} \mid \GET ra \mid \UPD rae \\
    O,P,Q &::= \PVAR \mid \NOTHING \mid \READ \mid \WRITE & \NOTHING & \sqsubseteq\WRITE\sqsubset \READ \\
    s,t &::= \TVAR \mid \NUM \mid s \to t \mid \REC{ \overline{a :^P t}} \\
    S, T& ::= \forall \TVAR. T \mid \forall \PVAR. T \mid t
  \end{align*}
  \caption{Syntax}
  \label{fig:syntax-3}
\end{figure}
In the syntax in Figure~\ref{fig:syntax-3}, effects are attached to single record labels, intuitively $P, Q \subseteq
\{ \READ, \WRITE \}$ but it does not make sense to distinguish
$\{\WRITE, \READ\}$ from $\{\WRITE\}$, so we take the three element lattice as indicated.

\begin{figure}[tp]
  \begin{mathpar}
    \inferrule{
      A = A_1 \OVERRIDE \dots \OVERRIDE A_n \\
      A_i \vdash e_i : t_i }{ A \vdash \REC{ \overline{a_i=e_i}} :
      \REC{ \overline{a_i :^{P_i} t_i}} }

    \inferrule { A \vdash e : \REC{ a:^\READ t \mid r}
      % \\ \READ \in P
    } { A \vdash \GET ea : t }

    \inferrule{
      A_0 \vdash e_0 : \REC{ a:^\WRITE t \mid r} \\
      A_1 \vdash e_1 : t \\
      % \WRITE \in P \\
      A = A_0 \OVERRIDE A_1 }{ A \vdash \UPD {e_0}a{e_1} : \REC{ a:^P
        t \mid r} }
    \\
    \inferrule{ }{ A, x: t \vdash x: t }

    \inferrule{
      A_0, x:t \vdash e : t' \\
      A = A_0^* }{ A \vdash \LAM x t e : (t \to t') }

  \inferrule[PT: Subtyping needed?]
  { A, x:s \vdash f :t  \\ s \SUB t}
  { A \vdash \FIX xtf : t}

  \inferrule{
    A_0 \vdash e_0 : (t_1 \to t_2) \\
    A_1 \vdash e_1 : t_1 \\
    A = A_0 \OVERRIDE A_1
  }{
    A \vdash \APP{e_0}e_1 : t_2
  }

  \inferrule
  { A_1 \vdash e_1 : t_1 \\
    A_2, x : t_1 \vdash e_2 : t_2 \\
    A = A_1 \OVERRIDE A_2
  }
  { A \vdash \LET x{e_1} e_2 : t}

  \inferrule
  {
    A = A_0 \OVERRIDE A_1 \\
    \JOIN{A_1}{A_2}{ A_3} \\
    \SPLIT{t_1}{  t_2}{ t_3} \\\\
    A_0 \vdash e : \NUM \\
    A_2 \vdash f : t_2 \\
    A_3 \vdash g : t_3
  }
  {
    A \vdash \IF e f g : t_1
  }
  % PT: is subtyping really needed? We are propagating demand information *backwards*, so (e.g.) the demand
  % on the branches of a conditional is the same as the demand on the result of the conditional.
  % \inferrule[Sub]
  % {A \vdash e : s \\ s \SUB t}
  % {A \vdash e : t}
  \end{mathpar}
  \caption{Typing rules.
  It seems like the subtyping in the FIX rule can alway be eliminated by unfolding the fixed point
  once.
  If the demand on $t$ is stronger or weaker than $s$, then unfolding will mediate the demand.
  NB the directionality of subtyping is wrong if interpolated from the lattice ordering (as it is presently).
}
  \label{fig:field-effects}
\end{figure}
% \begin{figure}[tp]
%   \begin{mathpar}
%     \inferrule{}{\NUM \SUB \NUM}
%
%     \inferrule{
%       s' \SUB s \\ t \SUB t'
%     }{
%       s \to t \SUB s' \to t' 
%     }
%
%     \inferrule{
%       \overline{s_i \SUB t_i} \\
%       \overline{P_i \sqsubseteq Q_i}
%     }{
%       \REC{ \overline{a_i :^{P_i} s_i}}
%       \SUB
%       \REC{ \overline{a_i :^{Q_i} t_i}}
%     }
%   \end{mathpar}
%   \caption{Subtyping}
%   \label{fig:subtyping-3}
% \end{figure}
\begin{figure}[tp]
  % \begin{align*}
  %   \NUM \sqcup \NUM
  %   & = \NUM
  %   \\
  %   (s \to t) \sqcup (s' \to t')
  %   & = (s \sqcap s') \to (t \sqcup t') \qquad ???
  %   \\
  %   \REC{ \overline{a_i :^{P_i} s_i}} \sqcup \REC{ \overline{a_i :^{Q_i} t_i}}
  %   &= \REC{ \overline{a_i :^{P_i \sqcup Q_i} (s_i \sqcup t_i)}}
  % \end{align*}
  \begin{mathpar}
    \inferrule{}{
      \SPLIT \NUM \NUM \NUM
    }

    \inferrule{}{
      \JOIN \NUM \NUM \NUM
    }
    \\
    \inferrule{
      \JOIN{s}{ s_1}{ s_2} \\
      \SPLIT{t}{t_1}{t_2}
    }{
      \SPLIT{s \to t}{s_1 \to t_1}{s_2 \to t_2}
    }

    \inferrule{
      \SPLIT{s}{ s_1}{ s_2} \\
      \JOIN{t}{t_1}{t_2}
    }{
      \JOIN{s \to t}{s_1 \to t_1}{s_2 \to t_2}
    }
    \\
    \inferrule
    {\overline{\SPLIT{s_i}{s_i'}{s_i''}}}
    {\SPLIT{\REC{ \overline{a_i :^{P_i} s_i}}}{\REC{ \overline{a_i :^{P_i} s_i'}}}{\REC{ \overline{a_i :^{P_i} s_i''}}}}

    \inferrule
    {\overline{\JOIN{s_i}{s_i'}{s_i''}} \\
    \overline{O_I = P_i \sqcup Q_i}}
    {\JOIN{\REC{ \overline{a_i :^{O_i} s_i}}}{\REC{ \overline{a_i :^{P_i} s_i'}}}{\REC{ \overline{a_i :^{Q_i} s_i''}}}}
  \end{mathpar}
  \caption{Bounding and splitting operations on types. The intuition is that \SPLITK{} propagates
    information in the positive positions of a type downwards towards the subexpressions. At the
    same time, it joins information in the negative positions and propagates the join upwards. The
    \JOINK{} operation works dually.}
  \label{fig:bounding-3}
\end{figure}
The novel ingredient in the typing rules are the propagation and splitting rules for the effect annotations on
record labels indicated by the ternary override relation $\OVERRIDE$.
Function closures have ``infinite'' splitting $(a :^P t)^*$ into $(a :^Q t)$.
\begin{itemize}
\item $P = \NOTHING$ propagate $Q=\NOTHING$.
\item $P = \READ$ propagate $Q = \WRITE$, no action required inside closure.
\item $P = \WRITE$ propagate $Q = \WRITE$, perform copy inside of closure.
\end{itemize}
 Propagating $\READ$ is not safe in both cases because that enables a write operation on the same
 record after the closure has been constructed. Alternatively, we could have another READONLY mode,
 which prevents subsequent write operations.

Function application has no particular restrictions, it seems.
Let improves on lambda combined with app because infinite splitting is avoided.

Look at combining $a :^O t \OVERRIDE a:^P t$ upwards to $a : ^Q t$:

$Q= O \OVERRIDE P$ is defined by:
\begin{itemize}
\item if $O=\NOTHING$ or $P=\NOTHING$ (field only used in one branch), then $Q = O \sqcup P$, no
  action required
\item if $O\sqsubseteq \READ$: $Q = O \sqcup P$,
  no further action required as potential writing in $P$ happens after reading
\item if $O = \WRITE$, $P = \READ$, then there are two choices
  \begin{itemize}
  \item $Q = \READ$ insert copy, propagate left to the write operation
  \item $Q = \WRITE$ insert copy, propagate right to the read operation
  \end{itemize}
  First option is preferred because it keeps writes local and prevents
  further copies downstream.
\item if $O=\WRITE$, $P =\WRITE$: $Q = \WRITE$ insert copy,
  propagate left (or right)
\end{itemize}
We need an operation that returns $\READ$ if a copy is required and
$\NOTHING$ otherwise.
\begin{align*}
  \COPY (O, P) &=
                 \begin{cases}
                   \READ & O=\WRITE \wedge \READ \sqsubseteq P \\
                   \NOTHING & \text{otherwise}
                 \end{cases}
  \\
  \COPY (\overline{O}, \overline{P})
               &= \bigsqcup_i \COPY (O_i, P_i)
  \\
  \COPY (\NUM, \NUM)
               & = \NOTHING
  \\
  \COPY (t_1 \to t_2, t_1 \to t_2)
               & = \NOTHING
  \\
  \COPY (\REC{ \overline{a_i :^{O_i} t_i'}}, \REC{ \overline{a_i :^{P_i} t_i''}})
  &=\COPY (\overline{O_i}, \overline{P_i})
    \sqcup \bigsqcup_i \overline{\COPY (t_i', t_i'')}
\end{align*}

Combining environments: pointwise

Combining types and records
\begin{mathpar}
  \inferrule{}{ \NUM = \NUM \OVERRIDE \NUM}

  \inferrule{}{t_1 \to t_2 = t_1 \to t_2 \OVERRIDE t_1 \to t_2}
  \\
  \inferrule{
    Q_i = (O_i \OVERRIDE P_i)
    \sqcup \COPY (\overline{O_i}, \overline{P_i})
    \sqcup \bigsqcup_i \overline{\COPY (t_i', t_i'')} \\
    t_i = t_i' \OVERRIDE t_i''
  }{
    \REC{ \overline{a_i :^{Q_i} t_i}}
    =
    \REC{ \overline{a_i :^{O_i} t_i'}}
    \OVERRIDE
    \REC{ \overline{a_i :^{P_i} t_i''}}
  }
\end{mathpar}

\begin{mathpar}
  \inferrule
  {}
  {
    \REC{ {a :^{Q_a} t_a}, {b :^{Q_b} t_b}}
    =
    \REC{ {a :^{O_a} t_a}, {b :^{O_b} t_b}}
    \OVERRIDE
    \REC{ {a :^{P_a} t_a}, {b :^{P_b} t_b}}
  }
\end{mathpar}

\clearpage
\begin{mathpar}
  \begin{array}{|c|c||c|c||c|c|l}
    Q_a&Q_b&O_a&O_b&P_a&P_b& \\
    \hline
    P_a &\NOTHING& \NOTHING & \NOTHING & P_a & \NOTHING & \\
    O_a & \NOTHING& O_a & \NOTHING & \NOTHING & \NOTHING & \\
    P_a\sqcup\READ   &\NOTHING& \READ & \NOTHING & P_a & \NOTHING & \\
    \READ&\READ& \WRITE & \NOTHING & \READ & \NOTHING
                       &  \text{copy, propagate left to write}\\
    \WRITE &\READ& \WRITE & \NOTHING & \WRITE & \NOTHING
                       &\text{copy, propagate left to write} \\
  \end{array}

  \begin{array}{|c|c||c|c||c|c|l}
    Q_a&Q_b&O_a&O_b&P_a&P_b& \\
    \hline
    P_a & P_b & \NOTHING & \NOTHING & P_a & P_b & \\
    O_a & P_b & O_a & \NOTHING & \NOTHING & P_b & \\
    P_a\sqcup\READ   &P_b& \READ & \NOTHING & P_a & P_b & \\
    \READ&P_b\sqcup\READ& \WRITE & \NOTHING & \READ & P_b
                       &  \text{copy, propagate left to write}\\
    \WRITE & P_b\sqcup\READ & \WRITE & \NOTHING & \WRITE & P_b
                       &\text{copy, propagate left to write} \\
  \end{array}

  \begin{array}{|c|c||c|c||c|c|l}
    Q_a&Q_b&O_a&O_b&P_a&P_b& \\
    \hline
    P_a & O_b & \NOTHING & O_b & P_a & \NOTHING & \\
    O_a & O_b & O_a & O_b & \NOTHING & \NOTHING & \\
    P_a\sqcup\READ  & O_b & \READ & O_b & P_a & \NOTHING & \\
    \READ&O_b \sqcup\READ& \WRITE & O_b & \READ & \NOTHING
                       &  \text{copy, propagate left to write}\\
    \WRITE &O_b\sqcup\READ& \WRITE & O_b & \WRITE & \NOTHING
                           &\text{copy, propagate left to write} \\
  \end{array}

\begin{array}{|c|c||c|c||c|c|l}
    Q_a&Q_b&O_a&O_b&P_a&P_b& \\
    \hline
    P_a &P_b\sqcup\READ& \NOTHING & \READ & P_a & P_b & \\
    O_a & P_b\sqcup\READ& O_a & \READ & \NOTHING & P_b & \\
    P_a\sqcup\READ   &P_b\sqcup\READ& \READ & \READ & P_a & P_b & \\
    \READ&P_b\sqcup\READ& \WRITE & \READ & \READ & P_b
                       &  \text{copy, propagate left to write}\\
    \WRITE &P_b\sqcup\READ& \WRITE & \READ & \WRITE & P_b
                       &\text{copy, propagate left to write} \\
  \end{array}

  \begin{array}{|c|c||c|c||c|c|l}
    Q_a&Q_b&O_a&O_b&P_a&P_b& \\
    \hline
    P_a\sqcup\READ &\READ& \NOTHING & \WRITE & P_a & \READ
                           & \text{copy, propagate left to write} \\
    \READ & \READ& O_a & \WRITE & \NOTHING & \READ
                           & \text{copy, propagate left to write} \\
    P_a\sqcup\READ   &\READ& \READ & \WRITE & P_a & \READ
                           &  \text{copy, propagate left to write}\\
    \READ&\READ& \WRITE & \WRITE & \READ & \READ
                       &  \text{copy, propagate left to write}\\
    \WRITE &\READ& \WRITE & \WRITE & \WRITE & \READ
                       &\text{copy, propagate left to write} \\
  \end{array}

  \begin{array}{|c|c||c|c||c|c|l}
    Q_a&Q_b&O_a&O_b&P_a&P_b& \\
    \hline
    P_a\sqcup\READ &\WRITE& \NOTHING & \WRITE & P_a & \WRITE
                           & \text{copy, propagate left to write} \\
    \READ & \WRITE& O_a & \WRITE & \NOTHING & \WRITE
                           & \text{copy, propagate left to write} \\
    P_a\sqcup\READ   &\WRITE& \READ & \WRITE & P_a & \WRITE
                           &  \text{copy, propagate left to write}\\
    \READ&\WRITE& \WRITE & \WRITE & \READ & \WRITE
                       &  \text{copy, propagate left to write}\\
    \WRITE &\WRITE& \WRITE & \WRITE & \WRITE & \WRITE
                       &\text{copy, propagate left to write} \\
  \end{array}
\end{mathpar}
A copy operation reads \textbf{every} field. Hence, a copy operation
caused by one field implies a read effect on all fields.
\clearpage{}
Consider the typing for 
\begin{align*}
  f
  &=(\LAM r{\REC{ a:^\WRITE \NUM}} \UPD r a{42})
  &&: \REC{ a:^\WRITE \NUM} \to \REC{ a:^\PVAR \NUM}
  \\
  g
  &=(\LAM r{\REC{a :^\NOTHING \NUM}} \REC{a = 0})
  &&: \REC{a :^\NOTHING \NUM} \to \REC{ a:^\PVAR \NUM}
  \\
  &\IF e{f}{g}
  &&: \REC{ a:^\WRITE \NUM} \to \REC{ a:^\PVAR \NUM}
\end{align*}
This propagation using $\WRITE\sqcup\NOTHING = \WRITE$ is correct,
\textbf{but} the propagation of $\WRITE$  may cause a useless copy if $e$ is false.

The typing is analogous to pushing the conditional into the lamdas.
\begin{align*}
  r : \REC{ a:^\WRITE \NUM} & \vdash \UPD r a{42} &&: \REC{ a:^\PVAR \NUM}
  \\
  r : \REC{ a:^\NOTHING \NUM} & \vdash \REC{a = 0} &&: \REC{ a:^\PVAR \NUM}
  \\
  r : \REC{ a:^\WRITE \NUM}& \vdash \IF e{( \UPD r a{42})}{(\REC{a = 0})} &&: \REC{ a:^\PVAR \NUM}
  \\
  \vdash \LAM r {\REC{ a:^\WRITE \NUM}} & \IF e{( \UPD r a{42})}{(\REC{a = 0})}
                                                  &&: {\REC{ a:^\WRITE \NUM}} \to \REC{ a:^\PVAR \NUM}
\end{align*}

\textbf{Let's finally get down to polymorphism. }
\begin{align*}
  \mathit{id} =
  \LAM x\TVAR x & : \forall\TVAR.\TVAR\to\TVAR
  &&\text{seems fine, just instantiate}
  \\
  \mathit{dup} =
  \LAM x\TVAR x & : \forall\TVAR.\TVAR\to(\TVAR,\TVAR)
\end{align*}
\begin{verbatim}
lambda r.
  let (r1, r2) = dup r 
  in  (r1 { a = 42 }, r2.a)
-- should behave like "write before read"
lambda r. 
      (r  { a = 42 }, r .a)
\end{verbatim}
\begin{align*}
  r : \REC{ a:^\WRITE \NUM} & \vdash \UPD r a{42} &&: \REC{ a:^\PVAR \NUM}
  \\
  r : \REC{ a:^\READ \NUM} & \vdash  r. a &&: \NUM
  \\
  r : \REC{ a:^\READ \NUM} & \vdash (\UPD r a{42}, r. a)  &&: \REC{ a:^\PVAR \NUM} \times \NUM
\end{align*}
Using \textit{dup}
\begin{align*}
  r_1 : \REC{ a:^\WRITE \NUM} & \vdash \UPD r a{42} &&: \REC{ a:^\PVAR \NUM}
  \\
  r_2 : \REC{ a:^\READ \NUM} & \vdash  r. a &&: \NUM
  \\
  r_1 : \REC{ a:^\WRITE \NUM},
  r_2 : \REC{ a:^\READ \NUM} & \vdash  (\UPD r a{42} , r. a) &&: \REC{ a:^\PVAR \NUM} \times \NUM
\end{align*}
\textbf{Suggestion}
Distiguish positive and negative occurrences of a type variable.
\begin{itemize}
\item If a negative occurrence of a type variable is instantiated with a record type, then its fields are all set to
  \READ.
\item Every instantiation of a positive occurrence of a type variable with a record type is allowed
  to have \textbf{different field permissions}. If a record type ends up with permission $\WRITE$,
  then the corresponding record is copied after instantiation.
\item If there is only one positive occurrence under an affine type constructor, then its permission
  is pushed through.
\end{itemize}
That does not quite solve the problem:
\begin{verbatim}
let (r1, r2) = (r, r)
in  (r1 { a = 42 }, r2.a)
\end{verbatim}
Here the reads and writes to \texttt{r1} and \texttt{r2} get out of order due to aliasing in the
pair.

Practical instance: the state monad. Turns out that the bind cannot be handled satisfactorily by the
suggestion on the table because we don't know that \texttt{s} is linear. Seems like linearity is
useful knowledge for polymorphic arguments.
\begin{verbatim}
type ST s a = s -> (s,a)

-- a: -, +
-- s: -, +
returnST :: a -> ST s a
returnST a = \s -> (s, a)

-- b: -, +
-- a: -, +
-- s: +, -, +, -, -, +
bindST :: ST s a -> (a -> ST s b) -> ST s b
bindST ma f = \s -> let (s1,a) = ma s 
                    in  f s1 a
\end{verbatim}
\end{document}

